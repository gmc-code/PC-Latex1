\documentclass{article}
\usepackage{tikz}
\usetikzlibrary{angles,quotes}
\usetikzlibrary{intersections}

% Define variables for the angles, line lengths, and distance between lines
\def\angle{54}
\def\linelength{4}
\def\paralleldistance{2}
\def\transangle{0}
\def\transprojectinglength{1.5}
\def\rotationAngle{45}  %{90-\angle}
% Calculate the translength
\pgfmathsetmacro{\translength}{2*\transprojectinglength + \paralleldistance/cos((90-\angle) + \transangle)}

% Calculate the x and y components of the first parallel line starting at 0, 0\
\pgfmathsetmacro{\parIstartx}{0}
\pgfmathsetmacro{\parIstarty}{0}
\pgfmathsetmacro{\parIendx}{\linelength*cos(\angle)}
\pgfmathsetmacro{\parIendy}{\linelength*sin(\angle)}

% Calculate the x and y offsets for the second line; use x shift only
\pgfmathsetmacro{\xoffset}{\paralleldistance/sin(\angle)}
\pgfmathsetmacro{\yoffset}{0}

% Calculate the x and y components of the second parallel line starting at 0, 0\
\pgfmathsetmacro{\parIIstartx}{0 + \xoffset}
\pgfmathsetmacro{\parIIstarty}{0 + \yoffset}
\pgfmathsetmacro{\parIIendx}{\parIendx + \xoffset}
\pgfmathsetmacro{\parIIendy}{\parIendy + \yoffset}

% Calculate the x and y components of the transversal vector
\pgfmathsetmacro{\transparIendx}{0.5*\translength*cos(\transangle)}
\pgfmathsetmacro{\transparIendy}{0.5*\translength*sin(\transangle)}

% Calculate the midpoint of the parallel lines
\pgfmathsetmacro{\midpointx}{0.5*\parIendx + 0.5*\xoffset}
\pgfmathsetmacro{\midpointy}{0.5*\parIendy}

% Calculate the start and end points of the transversal
\pgfmathsetmacro{\transstartx}{\midpointx -1* \transparIendx}
\pgfmathsetmacro{\transstarty}{\midpointy -1* \transparIendy}
\pgfmathsetmacro{\transendx}{\midpointx + \transparIendx}
\pgfmathsetmacro{\transendy}{\midpointy + \transparIendy}


% Draw the diagram
\begin{document}
    \begin{tikzpicture}
    % \tikzset{dot/.style={circle,inner sep=1pt,fill,label={#1},name=#1}}
    % Draw a grey 1cm grid behind the diagram
    % \draw[step=1cm,gray,very thin] (-1,-1) grid (6,4);
    
      \begin{scope}[rotate=\rotationAngle]
        % Draw the first line
        \draw[<->>, name path=P1]  (0,0) -- (\parIendx, \parIendy);
        
        % Draw the second line with the calculated offsets
        \draw[<->>, name path=P2] (\parIIstartx, \parIIstarty) -- (\parIIendx, \parIIendy);
        
        % Draw the transversal through the midpoint of one of the lines
        \draw[<->, name path=P3] (\transstartx, \transstarty) -- (\transendx, \transendy);

        \path [name intersections={of=P1 and P3,by=A}];
        % \node [dot=A] at (A) {};
        \path [name intersections={of=P2 and P3,by=B}];
        % \node [dot=B] at (B) {};

        % Draw the angle
        \coordinate (p1s) at (0,0);
        \coordinate (p1e) at (\parIendx, \parIendy);
        \coordinate (p2s) at (0,0);
        \coordinate (p2e) at (\parIIendx, \parIIendy);
        \coordinate (ts) at (\transstartx, \transstarty);
        \coordinate (te) at (\transendx, \transendy);
        
        \draw pic["$\alpha$", draw=red, -, angle eccentricity=1.4, angle radius=0.8cm] {angle=B--A--p1e};
        
         \draw pic["$\alpha$", draw=red, -, angle eccentricity=1.4, angle radius=0.8cm] {angle=te--B--p2e};
         
      \end{scope}
    \end{tikzpicture}
\end{document}
