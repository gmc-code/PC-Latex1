\documentclass[leqno, 12pt]{article}
\usepackage[a4paper, portrait, margin=2cm]{geometry}
\usepackage{amsmath}
\usepackage{multicol}
\usepackage{dcolumn}
    
\newcolumntype{d}[1]{D{.}{.}{#1}}
\usepackage{multicol}
\usepackage{fancyhdr}


\def \HeadingQuestions {\section*{\Large Name: \underline{\hspace{8cm}} \hfill Date: \underline{\hspace{3cm}}} \vspace{-3mm}
{Recurring decimals: Questions} \vspace{1pt}\hrule}

% raise footer with page number
\fancypagestyle{myfancypagestyle}{
  \fancyhf{} % clear all header and footer fields
  \renewcommand{\headrulewidth}{0pt} % no rule under header
  \fancyfoot[C] {\thepage} \setlength{\footskip}{16pt} % raise page number 6pt
}
\pagestyle{myfancypagestyle}  % apply myfancypagestyle

% 
\begin{document}
    \HeadingQuestions
    \vspace{-5mm}
    \begin{multicols}{3}
        \begin{flalign} 
    &\frac{1}{3} =\quad&
\end{flalign}



\vspace{12pt}\begin{flalign} 
    &\frac{2}{3} =\quad&
\end{flalign}



\vspace{12pt}\begin{flalign} 
    &\frac{1}{6} =\quad&
\end{flalign}



\vspace{12pt}\begin{flalign} 
    &\frac{5}{6} =\quad&
\end{flalign}



\vspace{12pt}\begin{flalign} 
    &\frac{1}{7} =\quad&
\end{flalign}



\vspace{12pt}\begin{flalign} 
    &\frac{2}{7} =\quad&
\end{flalign}



\vspace{12pt}\begin{flalign} 
    &\frac{3}{7} =\quad&
\end{flalign}



\vspace{12pt}\begin{flalign} 
    &\frac{4}{7} =\quad&
\end{flalign}



\vspace{12pt}\begin{flalign} 
    &\frac{5}{7} =\quad&
\end{flalign}



\vspace{12pt}\begin{flalign} 
    &\frac{6}{7} =\quad&
\end{flalign}



\vspace{12pt}\begin{flalign} 
    &\frac{1}{9} =\quad&
\end{flalign}



\vspace{12pt}\begin{flalign} 
    &\frac{2}{9} =\quad&
\end{flalign}



\vspace{12pt}\begin{flalign} 
    &\frac{4}{9} =\quad&
\end{flalign}



\vspace{12pt}\begin{flalign} 
    &\frac{5}{9} =\quad&
\end{flalign}



\vspace{12pt}\begin{flalign} 
    &\frac{7}{9} =\quad&
\end{flalign}



\vspace{12pt}\begin{flalign} 
    &\frac{8}{9} =\quad&
\end{flalign}



\vspace{12pt}\begin{flalign} 
    &\frac{1}{11} =\quad&
\end{flalign}



\vspace{12pt}\begin{flalign} 
    &\frac{2}{11} =\quad&
\end{flalign}



\vspace{12pt}\begin{flalign} 
    &\frac{3}{11} =\quad&
\end{flalign}



\vspace{12pt}\begin{flalign} 
    &\frac{4}{11} =\quad&
\end{flalign}



\vspace{12pt}\begin{flalign} 
    &\frac{5}{11} =\quad&
\end{flalign}



\vspace{12pt}\begin{flalign} 
    &\frac{6}{11} =\quad&
\end{flalign}



\vspace{12pt}\begin{flalign} 
    &\frac{7}{11} =\quad&
\end{flalign}



\vspace{12pt}\begin{flalign} 
    &\frac{8}{11} =\quad&
\end{flalign}



\vspace{12pt}\begin{flalign} 
    &\frac{9}{11} =\quad&
\end{flalign}



\vspace{12pt}\begin{flalign} 
    &\frac{10}{11} =\quad&
\end{flalign}



\vspace{12pt}
    \end{multicols}
\end{document}
