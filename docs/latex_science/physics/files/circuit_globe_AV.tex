\documentclass[12pt, varwidth, border=5mm]{standalone}
\usepackage{circuitikz}
\usepackage{tikz}
\usetikzlibrary{calc}

% define a custom ammeter component with overlaid text label
\newcommand{\myammeter}[2]{
    \draw (#1) to[rmeter] (#2);
    \node[align=center] at ($(#1)!0.5!(#2)$){A};
}

% define a custom voltmeter component with overlaid text label
\newcommand{\myvoltmeter}[2]{
    \draw (#1) to[rmeter] (#2);
    \node[align=center] at ($(#1)!0.5!(#2)$){V};
}

\begin{document}
\begin{circuitikz}
% define coordinates for points in circuit form bot left of main loop
\coordinate (A) at (0,0);
\coordinate (B) at (0,2);
\coordinate (C) at (3,2);
\coordinate (D) at (3,0);
\coordinate (E) at (0.5,0);
\coordinate (F) at (0.5,-2);
\coordinate (G) at (2.5,-2);
\coordinate (H) at (2.5,0);

% use the custom ammeter and voltmeter components
\draw
(A) to[battery] (B);
\myammeter{B}{C};
\draw
(C) -- (D)
(D) to[bulb] (A)
(E) -- (F);
\myvoltmeter{F}{G};
\draw
(G) -- (H)
;
 \end{circuitikz}\\
Series circuit with one globe and an ammeter and voltmeter.
\end{document}
